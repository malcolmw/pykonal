\section{Method}
	We implement the Fast Marching Method (FMM) for solving the Eikonal equation in 3D cartesian coordinates (Figure \ref{fig:coordinate_axes}) using a mixed- (first- and second-) order update scheme.
	\par
	
	\begin{figure}
	\centering
	\begin{tikzpicture}[axis/.style={thin, black, ->, >=stealth'}]
		\draw[axis] (0,0) -- (0,2) node [above, black] {\scriptsize -Y};
		\draw[axis] (0,0) -- (0,-2) node [below, black] {\scriptsize +Y};
		\draw[axis] (0,0) -- (2,0) node [right, black] {\scriptsize +X};
		\draw[axis] (0,0) -- (-2,0) node [left, black] {\scriptsize -X};
		\draw[axis] (0,0) -- (45:2) node [right, black] {\scriptsize +Z};
		\draw[axis] (0,0) -- (225:2) node [left, black] {\scriptsize -Z};
	\end{tikzpicture}
	\caption{Convention adopted throughout this article for the orientation of Cartesian coordinate axes}
	\label{fig:coordinate_axes}
\end{figure}

	The arrival time, $u$, of a front that moves perpendicular to itself is related to the propagation velocity, $v$, by the Eikonal equation, Eqn (\ref{eqn:eikonal}):
	
	\begin{equation}
		\label{eqn:eikonal}
		\left|\nabla u\left(\mathbf{r}\right)\right| = \frac{1}{v\left(\mathbf{r}\right)}
	\end{equation}
	
	\noindent where $\mathbf{r}$ is the position vector. Discretizing Eqn (\ref{eqn:eikonal}) gives
	
	\begin{equation}
		\label{eqn:discrete_eikonal}
		\left|\nabla u\left(i\Delta x, j\Delta y, k\Delta z\right)\right| = \frac{1}{v\left(i\Delta x, j\Delta y, k\Delta z\right)},
	\end{equation}
	\noindent	where $i, j, k \in \mathbb{Z}$ and $\Delta x, D\Delta y, \Delta z$ are the node intervals along the x, y, and z axis, respectively. Using index notation, i.e.,
	
	\begin{equation}
		f_{ijk} = f\left(i\Delta x, j\Delta y, k\Delta z\right),
	\end{equation}
	
	\noindent Eqn (\ref{eqn:discrete_eikonal}) becomes
	
	\begin{equation}
		\label{eqn:discrete_eikonal_index_form}
			\left|\nabla u_{ijk}\right| = \frac{1}{v_{ijk}}.
	\end{equation}
	
	Following Eqn [8] from \citeA{Sethian1996} and the reference to \citeA{Rouy1992} therein, we use the following discrete approximation to the gradient in the LHS of Eqn (\ref{eqn:discrete_eikonal_index_form}),
	
	\begin{equation}
		\left|\nabla u_{ijk}\right| ^2 \approx 
		\sum_{\xi \in \left[x, y, z\right]} max\left(D^{-\xi}_{ijk}u_{ijk}, -D^{+\xi}_{ijk}u_{ijk}, 0 \right)^2,
	\end{equation}
	
	\noindent which implies
	
	\begin{equation}
		\sum_{\xi \in \left[x, y, z\right]} max\left(D^{-\xi}_{ijk}u_{ijk}, -D^{+\xi}_{ijk}u_{ijk}, 0 \right)^2 - \frac{1}{v^2_{ijk}} \approx 0,
	\end{equation}
	
	where $D^{-\xi}_{ijk}$ and $D^{+\xi}_{ijk}$ represent backward and forward finite-difference operators along the $\xi$ axis, respectively. 
%	 and expanding the LHS of Eqn (\ref{eqn:discrete_eikonal_index_form}) yields
%	 
%	 \begin{equation}
%	 	\label{eqn:discrete_eikonal_expanded}
%		 \left(
%			   \frac{\partial u_{ijk}}{\partial x} 
%	 		+ \frac{\partial u_{ijk}}{\partial y}
%	 		+ \frac{\partial u_{ijk}}{\partial z}
%	 	\right)^2 = \frac{1}{v^2_{ijk}}.
%	\end{equation}
%	
%	\noindent Eqn (\ref{eqn:discrete_eikonal_expanded}) can be approximated using finite differences:
%	
%	\begin{equation}
%		\label{eqn:discrete_eikonal_approximate}
%		\left(D_x u_{ijk} + D_y u_{ijk} + D_z u_{ijk}\right)^2 = \frac{1}{v^2_{ijk}}.
%	\end{equation}
%	
%	\noindent where $D_\xi$ represents the finite difference with respect to the $\xi$ axis.
	\par